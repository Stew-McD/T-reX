%% Generated by Sphinx.
\def\sphinxdocclass{report}
\documentclass[letterpaper,10pt,english]{sphinxmanual}
\ifdefined\pdfpxdimen
   \let\sphinxpxdimen\pdfpxdimen\else\newdimen\sphinxpxdimen
\fi \sphinxpxdimen=.75bp\relax
\ifdefined\pdfimageresolution
    \pdfimageresolution= \numexpr \dimexpr1in\relax/\sphinxpxdimen\relax
\fi
%% let collapsible pdf bookmarks panel have high depth per default
\PassOptionsToPackage{bookmarksdepth=5}{hyperref}

\PassOptionsToPackage{booktabs}{sphinx}
\PassOptionsToPackage{colorrows}{sphinx}

\PassOptionsToPackage{warn}{textcomp}
\usepackage[utf8]{inputenc}
\ifdefined\DeclareUnicodeCharacter
% support both utf8 and utf8x syntaxes
  \ifdefined\DeclareUnicodeCharacterAsOptional
    \def\sphinxDUC#1{\DeclareUnicodeCharacter{"#1}}
  \else
    \let\sphinxDUC\DeclareUnicodeCharacter
  \fi
  \sphinxDUC{00A0}{\nobreakspace}
  \sphinxDUC{2500}{\sphinxunichar{2500}}
  \sphinxDUC{2502}{\sphinxunichar{2502}}
  \sphinxDUC{2514}{\sphinxunichar{2514}}
  \sphinxDUC{251C}{\sphinxunichar{251C}}
  \sphinxDUC{2572}{\textbackslash}
\fi
\usepackage{cmap}
\usepackage[T1]{fontenc}
\usepackage{amsmath,amssymb,amstext}
\usepackage{babel}



\usepackage{tgtermes}
\usepackage{tgheros}
\renewcommand{\ttdefault}{txtt}



\usepackage[Bjarne]{fncychap}
\usepackage{sphinx}

\fvset{fontsize=auto}
\usepackage{geometry}


% Include hyperref last.
\usepackage{hyperref}
% Fix anchor placement for figures with captions.
\usepackage{hypcap}% it must be loaded after hyperref.
% Set up styles of URL: it should be placed after hyperref.
\urlstyle{same}

\addto\captionsenglish{\renewcommand{\contentsname}{Contents:}}

\usepackage{sphinxmessages}
\setcounter{tocdepth}{1}



\title{WasteAndMaterialFootprint Documentation}
\date{Dec 26, 2023}
\release{0.1.2}
\author{Stewart Charles McDowall}
\newcommand{\sphinxlogo}{\vbox{}}
\renewcommand{\releasename}{Release}
\makeindex
\begin{document}

\ifdefined\shorthandoff
  \ifnum\catcode`\=\string=\active\shorthandoff{=}\fi
  \ifnum\catcode`\"=\active\shorthandoff{"}\fi
\fi

\pagestyle{empty}
\sphinxmaketitle
\pagestyle{plain}
\sphinxtableofcontents
\pagestyle{normal}
\phantomsection\label{\detokenize{index::doc}}


\sphinxstepscope


\chapter{Introduction}
\label{\detokenize{intro:introduction}}\label{\detokenize{intro::doc}}
\sphinxAtStartPar
The WasteAndMaterialFootprint tool is a python package that allows one to calculate the waste and material footprint of any product or service inside of the life cycle assessment database ecoinvent. The tool is based on the paper

\sphinxAtStartPar
\sphinxstylestrong{* THESE DOCUMENTS ARE STILL UNDER CONSTRUCTION *}

\sphinxAtStartPar
The full api reference is available on this site, however


\section{Motivation}
\label{\detokenize{intro:motivation}}

\section{Limitations}
\label{\detokenize{intro:limitations}}
\sphinxstepscope


\chapter{Installation}
\label{\detokenize{installation:installation}}\label{\detokenize{installation::doc}}

\section{Dependencies}
\label{\detokenize{installation:dependencies}}
\sphinxAtStartPar
The program is written in Python and the required packages are listed in the \sphinxtitleref{requirements.txt} file. These should be installed automatically when installing the program.

\sphinxAtStartPar
The main dependencies are:
\begin{itemize}
\item {} 
\sphinxAtStartPar
\sphinxhref{https://docs.brightway.dev}{brightway2}

\item {} 
\sphinxAtStartPar
\sphinxhref{https://premise.readthedocs.io}{premise}

\item {} 
\sphinxAtStartPar
\sphinxhref{https://wurst.readthedocs.io}{wurst}

\end{itemize}


\section{Installation Instructions}
\label{\detokenize{installation:installation-instructions}}
\sphinxAtStartPar
\sphinxstylestrong{It is recommended to use a fresh virtual environment to install the program.}

\sphinxAtStartPar
You can simply clone the repo and run:

\begin{sphinxVerbatim}[commandchars=\\\{\}]
python\PYG{+w}{ }src/WasteAndMaterialFootprint/main.py
\end{sphinxVerbatim}

\sphinxAtStartPar
This will not install any of the dependencies, so you will need to install them manually if you don’t already have them.

\sphinxAtStartPar
A better option: the program can be installed using pip:

\begin{sphinxVerbatim}[commandchars=\\\{\}]
pip\PYG{+w}{ }install\PYG{+w}{ }WasteAndMaterialFootprint
\end{sphinxVerbatim}

\sphinxAtStartPar
Or, if you want to install the latest version from GitHub:

\begin{sphinxVerbatim}[commandchars=\\\{\}]
pip\PYG{+w}{ }install\PYG{+w}{ }git+https://github.com/Stew\PYGZhy{}McD/WasteAndMaterialFootprint.git
\end{sphinxVerbatim}

\sphinxAtStartPar
Or for an editable install (good for development and testing):

\begin{sphinxVerbatim}[commandchars=\\\{\}]
git\PYG{+w}{ }clone\PYG{+w}{ }https://github.com/Stew\PYGZhy{}McD/WasteAndMaterialFootprint.git
\PYG{n+nb}{cd}\PYG{+w}{ }WasteAndMaterialFootprint
pip\PYG{+w}{ }install\PYG{+w}{ }\PYGZhy{}e\PYG{+w}{ }.
\end{sphinxVerbatim}

\sphinxstepscope


\chapter{Usage}
\label{\detokenize{usage:usage}}\label{\detokenize{usage::doc}}
\sphinxAtStartPar
The program can be used directly from the command line, or imported as a Python module. This will run the program using the default settings. See the \DUrole{xref,std,std-ref}{configuration} section for more information on how to change the settings.


\section{Command Line}
\label{\detokenize{usage:command-line}}
\sphinxAtStartPar
You should clone the repo, navigate to the \sphinxcode{\sphinxupquote{WasteAndMaterialFootprint}} folder, and then run the program using:

\begin{sphinxVerbatim}[commandchars=\\\{\}]
python\PYG{+w}{ }src/WasteAndMaterialFootprint/main.py
\end{sphinxVerbatim}


\section{Python Module}
\label{\detokenize{usage:python-module}}
\sphinxAtStartPar
The program can be imported as a Python module:

\begin{sphinxVerbatim}[commandchars=\\\{\}]
\PYG{k+kn}{import} \PYG{n+nn}{WasteAndMaterialFootprint} \PYG{k}{as} \PYG{n+nn}{wmf}
\PYG{n}{wmf}\PYG{o}{.}\PYG{n}{run}\PYG{p}{(}\PYG{p}{)}
\end{sphinxVerbatim}

\sphinxstepscope


\chapter{Configuration}
\label{\detokenize{configuration:configuration}}\label{\detokenize{configuration::doc}}
\sphinxAtStartPar
By default, the program will create a folder \sphinxcode{\sphinxupquote{config}} in the current working directory containing the default configuration files:


\section{General Settings: \sphinxstyleliteralintitle{\sphinxupquote{user\_settings.py}}}
\label{\detokenize{configuration:general-settings-user-settings-py}}
\sphinxAtStartPar
This is the main configuration file, the one that you might want to edit to match your project structure and your needs. By default, the program will take a brightway2 project named \sphinxcode{\sphinxupquote{default}} and copy that to a new project named \sphinxcode{\sphinxupquote{SSP\sphinxhyphen{}cutoff}}, which is then copied to a new project named \sphinxcode{\sphinxupquote{WMFootprint\sphinxhyphen{}SSP\sphinxhyphen{}cutoff}}.

\sphinxAtStartPar
Doing it this way isolates the components and allows you to keep your original brightway2 project as it was. If space is an issue, you can set all of the project names to be the same.

\sphinxAtStartPar
If you are happy with the default settings, you can just run the program and it will create the databases for you. If you want to change the settings, you can edit the \sphinxcode{\sphinxupquote{user\_settings.py}} file that you can find in the \sphinxcode{\sphinxupquote{config}} directory of your working directory.

\sphinxAtStartPar
These are some extracts from \sphinxcode{\sphinxupquote{user\_settings.py}} with the most important settings (the ones you might want to change) and their default values:

\begin{sphinxVerbatim}[commandchars=\\\{\}]
\PYG{c+c1}{\PYGZsh{} Choose whether to use premise to create future scenario databases}
\PYG{n}{use\PYGZus{}premise} \PYG{o}{=} \PYG{k+kc}{True}
\PYG{c+c1}{\PYGZsh{} Choose whether to use WasteAndMaterialFootprint to edit the databases (you could also turn this off and just use the package as an easy way to make a set of future scenario databases)}
\PYG{n}{use\PYGZus{}wmf} \PYG{o}{=} \PYG{k+kc}{True}

\PYG{c+c1}{\PYGZsh{} Choose the names of the projects to use}
\PYG{n}{project\PYGZus{}premise\PYGZus{}base} \PYG{o}{=} \PYG{l+s+s2}{\PYGZdq{}}\PYG{l+s+s2}{default}\PYG{l+s+s2}{\PYGZdq{}}
\PYG{n}{project\PYGZus{}premise} \PYG{o}{=} \PYG{l+s+s2}{\PYGZdq{}}\PYG{l+s+s2}{SSP\PYGZhy{}cutoff}\PYG{l+s+s2}{\PYGZdq{}}
\PYG{n}{project\PYGZus{}base} \PYG{o}{=} \PYG{n}{project\PYGZus{}premise}
\PYG{n}{project\PYGZus{}wmf} \PYG{o}{=} \PYG{l+s+sa}{f}\PYG{l+s+s2}{\PYGZdq{}}\PYG{l+s+s2}{WMFootprint\PYGZhy{}}\PYG{l+s+si}{\PYGZob{}}\PYG{n}{project\PYGZus{}base}\PYG{l+s+si}{\PYGZcb{}}\PYG{l+s+s2}{\PYGZdq{}}

\PYG{c+c1}{\PYGZsh{} Choose the name of the database to use (needed for premise only, the WMF tool will run all databases except the biospheres)}
\PYG{n}{database\PYGZus{}name} \PYG{o}{=} \PYG{l+s+s2}{\PYGZdq{}}\PYG{l+s+s2}{ecoinvent\PYGZhy{}3.9.1\PYGZhy{}cutoff}\PYG{l+s+s2}{\PYGZdq{}}

\PYG{c+c1}{\PYGZsh{} if you want to use a fresh project}
\PYG{n}{delete\PYGZus{}existing\PYGZus{}premise\PYGZus{}project} \PYG{o}{=} \PYG{k+kc}{False}
\PYG{n}{delete\PYGZus{}existing\PYGZus{}wmf\PYGZus{}project} \PYG{o}{=} \PYG{k+kc}{False}

\PYG{c+c1}{\PYGZsh{} Choose the premise scenarios to generate (see FutureScenarios.py for more details)}
\PYG{c+c1}{\PYGZsh{} Not all combinations are available, the code in FutureScenarios.py will filter out the scenarios that are not possible}
\PYG{c+c1}{\PYGZsh{} the default is to have an optimistic and a pessimistic scenario with SSP2 for 2030, 2065 and 2100}

\PYG{n}{models} \PYG{o}{=} \PYG{p}{[}\PYG{l+s+s2}{\PYGZdq{}}\PYG{l+s+s2}{remind}\PYG{l+s+s2}{\PYGZdq{}}\PYG{p}{]}
\PYG{n}{ssps} \PYG{o}{=} \PYG{p}{[}\PYG{l+s+s2}{\PYGZdq{}}\PYG{l+s+s2}{SSP2}\PYG{l+s+s2}{\PYGZdq{}}\PYG{p}{]}
\PYG{n}{rcps} \PYG{o}{=} \PYG{p}{[}\PYG{l+s+s2}{\PYGZdq{}}\PYG{l+s+s2}{Base}\PYG{l+s+s2}{\PYGZdq{}}\PYG{p}{,}\PYG{l+s+s2}{\PYGZdq{}}\PYG{l+s+s2}{PkBudg500}\PYG{l+s+s2}{\PYGZdq{}}\PYG{p}{]}
\PYG{n}{years} \PYG{o}{=} \PYG{p}{[}\PYG{l+m+mi}{2030}\PYG{p}{,}\PYG{l+m+mi}{2065}\PYG{p}{,}\PYG{l+m+mi}{2100}\PYG{p}{,}\PYG{p}{]}
\end{sphinxVerbatim}


\section{Waste Search Settings: \sphinxstyleliteralintitle{\sphinxupquote{queries\_waste.py}}}
\label{\detokenize{configuration:waste-search-settings-queries-waste-py}}
\sphinxAtStartPar
This file sets up search parameters for different waste and material flow categories, crucial for the \sphinxcode{\sphinxupquote{SearchWaste.py}} script. It leverages a \sphinxcode{\sphinxupquote{.pickle}} file created by \sphinxcode{\sphinxupquote{ExplodeDatabase.py}}.


\subsection{Categories}
\label{\detokenize{configuration:categories}}
\sphinxAtStartPar
Handles various categories like digestion, composting, incineration, recycling, landfill, etc.


\subsection{Query Types}
\label{\detokenize{configuration:query-types}}
\sphinxAtStartPar
Two sets of queries are created:
\begin{enumerate}
\sphinxsetlistlabels{\arabic}{enumi}{enumii}{}{.}%
\item {} 
\sphinxAtStartPar
\sphinxcode{\sphinxupquote{queries\_kg}} for waste flows in kilograms.

\item {} 
\sphinxAtStartPar
\sphinxcode{\sphinxupquote{queries\_m3}} for waste flows in cubic meters.

\end{enumerate}


\subsection{Adjusting Search Terms}
\label{\detokenize{configuration:adjusting-search-terms}}\begin{itemize}
\item {} 
\sphinxAtStartPar
\sphinxstylestrong{Search Keywords}: Tweak the \sphinxcode{\sphinxupquote{AND}}, \sphinxcode{\sphinxupquote{OR}}, \sphinxcode{\sphinxupquote{NOT}} lists to refine your search.

\end{itemize}


\subsection{Category\sphinxhyphen{}Specific Changes}
\label{\detokenize{configuration:category-specific-changes}}\begin{itemize}
\item {} 
\sphinxAtStartPar
\sphinxstylestrong{Adding Categories}: You can add new categories to the \sphinxcode{\sphinxupquote{names}} list.

\item {} 
\sphinxAtStartPar
\sphinxstylestrong{Modifying Queries}: Update the query parameters for each category based on your requirements.

\end{itemize}


\subsection{Optimizing Search Efficiency}
\label{\detokenize{configuration:optimizing-search-efficiency}}
\sphinxAtStartPar
You can choose to include or exclude whatever you want. For instance, “non\sphinxhyphen{}hazardous” is not included as it’s derivable from other categories and slows down the process.


\subsection{Validating Search Terms}
\label{\detokenize{configuration:validating-search-terms}}
\sphinxAtStartPar
Isolate the function of \sphinxcode{\sphinxupquote{SearchWaste.py}} to validate your search terms. That means, turning off the other functions in \sphinxcode{\sphinxupquote{user\_settings.py}}, or running the module directly. You can achieve this by setting the following in \sphinxcode{\sphinxupquote{user\_settings.py}}:

\begin{sphinxVerbatim}[commandchars=\\\{\}]
\PYG{n}{use\PYGZus{}premise} \PYG{o}{=} \PYG{k+kc}{False}
\PYG{n}{do\PYGZus{}search} \PYG{o}{=} \PYG{k+kc}{True}
\PYG{n}{do\PYGZus{}methods} \PYG{o}{=} \PYG{k+kc}{False}
\PYG{n}{do\PYGZus{}edit} \PYG{o}{=} \PYG{k+kc}{False}
\end{sphinxVerbatim}


\section{Material Search Settings: \sphinxstyleliteralintitle{\sphinxupquote{queries\_materials.py}}}
\label{\detokenize{configuration:material-search-settings-queries-materials-py}}
\sphinxAtStartPar
The \sphinxcode{\sphinxupquote{queries\_materials}} module creates demand methods in the WasteAndMaterialFootprint tool. It aligns with the EU CRM list 2023 and the ecoinvent database, incorporating additional strategic materials for comprehensive analysis. More can be easily added, as wished by the user.

\sphinxAtStartPar
This function uses the string tests \sphinxcode{\sphinxupquote{startswith}} in \sphinxcode{\sphinxupquote{SearchMaterial.py}} to identify activities beginning with the specified material name. This allows one to be more specific with the search terms (the \sphinxcode{\sphinxupquote{,}} can be critical sometimes).


\subsection{Structure and Customisation}
\label{\detokenize{configuration:structure-and-customisation}}

\subsubsection{Tuple Structure}
\label{\detokenize{configuration:tuple-structure}}\begin{itemize}
\item {} 
\sphinxAtStartPar
\sphinxstylestrong{First Part (Activity Name)}: Specifies the exact activity in the database (e.g., \sphinxcode{\sphinxupquote{market for chromium}}).

\item {} 
\sphinxAtStartPar
\sphinxstylestrong{Second Part (Material Category)}: Aggregates related activities under a common category (e.g., \sphinxcode{\sphinxupquote{chromium}}), enhancing data processing efficiency.

\end{itemize}


\subsubsection{Customisation Options}
\label{\detokenize{configuration:customisation-options}}\begin{itemize}
\item {} 
\sphinxAtStartPar
\sphinxstylestrong{Add or Remove Materials}: Adapt the tuple list by including new materials or removing irrelevant ones.

\item {} 
\sphinxAtStartPar
\sphinxstylestrong{Refine Search Terms}: Update material categories for a better fit with your database, ensuring precision in naming, especially with the use of commas.

\end{itemize}


\subsection{Usage Considerations}
\label{\detokenize{configuration:usage-considerations}}\begin{itemize}
\item {} 
\sphinxAtStartPar
\sphinxstylestrong{Material Quantity}: The current list comprises over 40 materials. Modify this count to suit your project’s scope.

\item {} 
\sphinxAtStartPar
\sphinxstylestrong{Database Alignment}: Check that the material names correspond with your specific database version, like ecoinvent v3.9.1.

\end{itemize}


\subsubsection{Example Tuples}
\label{\detokenize{configuration:example-tuples}}\begin{itemize}
\item {} 
\sphinxAtStartPar
\sphinxcode{\sphinxupquote{("market for chromium", "chromium")}}

\item {} 
\sphinxAtStartPar
\sphinxcode{\sphinxupquote{("market for coal", "coal")}}

\item {} 
\sphinxAtStartPar
\sphinxcode{\sphinxupquote{("market for cobalt", "cobalt")}}

\item {} 
\sphinxAtStartPar
\sphinxcode{\sphinxupquote{("market for coke", "coke")}}

\item {} 
\sphinxAtStartPar
\sphinxcode{\sphinxupquote{("market for copper", "copper")}}

\item {} 
\sphinxAtStartPar
\sphinxcode{\sphinxupquote{("market for tap water", "water")}}

\item {} 
\sphinxAtStartPar
\sphinxcode{\sphinxupquote{("market for water,", "water")}}

\end{itemize}

\sphinxstepscope


\chapter{Examples}
\label{\detokenize{examples:examples}}\label{\detokenize{examples::doc}}
\sphinxAtStartPar
See the \sphinxtitleref{examples} directory for an examples of how to use the WasteAndMaterialFootprint package.
The folder \sphinxtitleref{batteries} contains a small case study using the package to calculate the waste and material footprints of several battery technologies in ecoinvent 3.10.

\sphinxstepscope


\chapter{API Reference}
\label{\detokenize{modules:api-reference}}\label{\detokenize{modules::doc}}
\sphinxstepscope


\section{WasteAndMaterialFootprint package}
\label{\detokenize{WasteAndMaterialFootprint:wasteandmaterialfootprint-package}}\label{\detokenize{WasteAndMaterialFootprint::doc}}

\subsection{WasteAndMaterialFootprint.main module}
\label{\detokenize{WasteAndMaterialFootprint:module-WasteAndMaterialFootprint.main}}\label{\detokenize{WasteAndMaterialFootprint:wasteandmaterialfootprint-main-module}}\index{module@\spxentry{module}!WasteAndMaterialFootprint.main@\spxentry{WasteAndMaterialFootprint.main}}\index{WasteAndMaterialFootprint.main@\spxentry{WasteAndMaterialFootprint.main}!module@\spxentry{module}}

\subsubsection{main Module}
\label{\detokenize{WasteAndMaterialFootprint:main-module}}
\sphinxAtStartPar
Main module of the \sphinxtitleref{WasteAndMaterialFootprint} tool.

\sphinxAtStartPar
This script serves as the entry point for the \sphinxtitleref{WasteAndMaterialFootprint} tool. It orchestrates the overall process, including the setup and execution of various subprocesses like database explosion, material and waste searches, and the editing of exchanges.

\sphinxAtStartPar
The script supports both single and multiple project/database modes, as well as the option to use multiprocessing. It also facilitates the use of the premise module to generate future scenario databases.


\paragraph{Customisation:}
\label{\detokenize{WasteAndMaterialFootprint:customisation}}\begin{itemize}
\item {} 
\sphinxAtStartPar
Project and database names, and other settings can be edited in \sphinxtitleref{config/user\_settings.py}.

\item {} 
\sphinxAtStartPar
Waste search query terms can be customised in \sphinxtitleref{config/queries\_waste.py}.

\item {} 
\sphinxAtStartPar
The list of materials can be modified in \sphinxtitleref{config/queries\_materials.py}.

\end{itemize}


\paragraph{Usage:}
\label{\detokenize{WasteAndMaterialFootprint:usage}}
\sphinxAtStartPar
To use the default settings, run the script with \sphinxtitleref{python main.py}. 
Arguments can be provided to change project/database names or to delete the project before running.
\index{EditExchanges() (in module WasteAndMaterialFootprint.main)@\spxentry{EditExchanges()}\spxextra{in module WasteAndMaterialFootprint.main}}

\begin{fulllineitems}
\phantomsection\label{\detokenize{WasteAndMaterialFootprint:WasteAndMaterialFootprint.main.EditExchanges}}
\pysigstartsignatures
\pysiglinewithargsret{\sphinxcode{\sphinxupquote{WasteAndMaterialFootprint.main.}}\sphinxbfcode{\sphinxupquote{EditExchanges}}}{\sphinxparam{\DUrole{n}{args}}}{}
\pysigstopsignatures
\sphinxAtStartPar
Edit exchanges in the database.

\sphinxAtStartPar
This function adds waste and material flows to the activities and verifies the database.
\begin{quote}\begin{description}
\sphinxlineitem{Parameters}
\sphinxAtStartPar
\sphinxstyleliteralstrong{\sphinxupquote{args}} \textendash{} Dictionary containing database and project settings.

\sphinxlineitem{Returns}
\sphinxAtStartPar
None

\end{description}\end{quote}

\end{fulllineitems}

\index{ExplodeAndSearch() (in module WasteAndMaterialFootprint.main)@\spxentry{ExplodeAndSearch()}\spxextra{in module WasteAndMaterialFootprint.main}}

\begin{fulllineitems}
\phantomsection\label{\detokenize{WasteAndMaterialFootprint:WasteAndMaterialFootprint.main.ExplodeAndSearch}}
\pysigstartsignatures
\pysiglinewithargsret{\sphinxcode{\sphinxupquote{WasteAndMaterialFootprint.main.}}\sphinxbfcode{\sphinxupquote{ExplodeAndSearch}}}{\sphinxparam{\DUrole{n}{args}}}{}
\pysigstopsignatures
\sphinxAtStartPar
Exploding the database into separate exchanges, searching for waste and
material flows, and processing these results.
\begin{description}
\sphinxlineitem{This includes:}\begin{itemize}
\item {} 
\sphinxAtStartPar
ExplodeDatabase.py

\item {} 
\sphinxAtStartPar
SearchWaste.py

\item {} 
\sphinxAtStartPar
SearchMaterial.py

\end{itemize}

\end{description}
\begin{quote}\begin{description}
\sphinxlineitem{Parameters}
\sphinxAtStartPar
\sphinxstyleliteralstrong{\sphinxupquote{args}} \textendash{} Dictionary containing database and project settings.

\sphinxlineitem{Returns}
\sphinxAtStartPar
None

\end{description}\end{quote}

\end{fulllineitems}

\index{run() (in module WasteAndMaterialFootprint.main)@\spxentry{run()}\spxextra{in module WasteAndMaterialFootprint.main}}

\begin{fulllineitems}
\phantomsection\label{\detokenize{WasteAndMaterialFootprint:WasteAndMaterialFootprint.main.run}}
\pysigstartsignatures
\pysiglinewithargsret{\sphinxcode{\sphinxupquote{WasteAndMaterialFootprint.main.}}\sphinxbfcode{\sphinxupquote{run}}}{}{}
\pysigstopsignatures
\sphinxAtStartPar
Main function serving as the wrapper for the WasteAndMaterialFootprint tool.
\begin{description}
\sphinxlineitem{This function coordinates the various components of the tool, including:}
\sphinxAtStartPar
creating future scenario databases,
setting up and processing each database for waste and material footprinting,
and combining results into a custom database.
adding LCIA methods to the project for each of the waste/material flows.

\end{description}

\sphinxAtStartPar
The function supports various modes of operation based on the settings in \sphinxtitleref{config/user\_settings.py}.
Specifications for material and waste searches can be customised in \sphinxtitleref{queries\_materials}.

\end{fulllineitems}



\subsection{Submodules}
\label{\detokenize{WasteAndMaterialFootprint:submodules}}

\subsection{WasteAndMaterialFootprint.ExchangeEditor module}
\label{\detokenize{WasteAndMaterialFootprint:module-WasteAndMaterialFootprint.ExchangeEditor}}\label{\detokenize{WasteAndMaterialFootprint:wasteandmaterialfootprint-exchangeeditor-module}}\index{module@\spxentry{module}!WasteAndMaterialFootprint.ExchangeEditor@\spxentry{WasteAndMaterialFootprint.ExchangeEditor}}\index{WasteAndMaterialFootprint.ExchangeEditor@\spxentry{WasteAndMaterialFootprint.ExchangeEditor}!module@\spxentry{module}}

\subsubsection{ExchangeEditor Module}
\label{\detokenize{WasteAndMaterialFootprint:exchangeeditor-module}}
\sphinxAtStartPar
This module is responsible for editing exchanges with wurst and Brightway2.
It appends relevant exchanges from the \sphinxtitleref{db\_wmf} (database containing waste and material exchange details) 
to activities identified by \sphinxtitleref{WasteAndMaterialSearch()} in the specified project’s database (\sphinxtitleref{db\_name}).
Each appended exchange replicates the same amount and unit as the original technosphere waste and material exchange.
\index{ExchangeEditor() (in module WasteAndMaterialFootprint.ExchangeEditor)@\spxentry{ExchangeEditor()}\spxextra{in module WasteAndMaterialFootprint.ExchangeEditor}}

\begin{fulllineitems}
\phantomsection\label{\detokenize{WasteAndMaterialFootprint:WasteAndMaterialFootprint.ExchangeEditor.ExchangeEditor}}
\pysigstartsignatures
\pysiglinewithargsret{\sphinxcode{\sphinxupquote{WasteAndMaterialFootprint.ExchangeEditor.}}\sphinxbfcode{\sphinxupquote{ExchangeEditor}}}{\sphinxparam{\DUrole{n}{project\_wmf}}\sphinxparamcomma \sphinxparam{\DUrole{n}{db\_name}}\sphinxparamcomma \sphinxparam{\DUrole{n}{db\_wmf\_name}}}{}
\pysigstopsignatures
\sphinxAtStartPar
Append relevant exchanges from \sphinxtitleref{db\_wmf} to each activity in \sphinxtitleref{db\_name} identified by \sphinxtitleref{WasteAndMaterialSearch()}.

\sphinxAtStartPar
This function modifies the specified project’s database by appending exchanges from the \sphinxtitleref{db\_wmf} to activities identified by \sphinxtitleref{WasteAndMaterialSearch()}. The appended exchanges mirror the quantity and unit of the original technosphere waste and material exchange.
\begin{quote}\begin{description}
\sphinxlineitem{Parameters}\begin{itemize}
\item {} 
\sphinxAtStartPar
\sphinxstyleliteralstrong{\sphinxupquote{project\_wmf}} (\sphinxstyleliteralemphasis{\sphinxupquote{str}}) \textendash{} Name of the Brightway2 project to be modified.

\item {} 
\sphinxAtStartPar
\sphinxstyleliteralstrong{\sphinxupquote{db\_name}} (\sphinxstyleliteralemphasis{\sphinxupquote{str}}) \textendash{} Name of the database within the project where activities and exchanges are stored.

\item {} 
\sphinxAtStartPar
\sphinxstyleliteralstrong{\sphinxupquote{db\_wmf\_name}} (\sphinxstyleliteralemphasis{\sphinxupquote{str}}) \textendash{} Name of the database containing waste and material exchange details.

\end{itemize}

\sphinxlineitem{Returns}
\sphinxAtStartPar
None. Modifies the given Brightway2 project by appending exchanges and logs statistics about the added exchanges.

\sphinxlineitem{Return type}
\sphinxAtStartPar
None

\sphinxlineitem{Raises}
\sphinxAtStartPar
\sphinxstyleliteralstrong{\sphinxupquote{Exception}} \textendash{} If any specified process or exchange is not found in the database.

\end{description}\end{quote}

\end{fulllineitems}



\subsection{WasteAndMaterialFootprint.ExplodeDatabase module}
\label{\detokenize{WasteAndMaterialFootprint:module-WasteAndMaterialFootprint.ExplodeDatabase}}\label{\detokenize{WasteAndMaterialFootprint:wasteandmaterialfootprint-explodedatabase-module}}\index{module@\spxentry{module}!WasteAndMaterialFootprint.ExplodeDatabase@\spxentry{WasteAndMaterialFootprint.ExplodeDatabase}}\index{WasteAndMaterialFootprint.ExplodeDatabase@\spxentry{WasteAndMaterialFootprint.ExplodeDatabase}!module@\spxentry{module}}

\subsubsection{ExplodeDatabase Module}
\label{\detokenize{WasteAndMaterialFootprint:explodedatabase-module}}
\sphinxAtStartPar
This module is responsible for exploding a Brightway2 database into a single\sphinxhyphen{}level list of all exchanges.
It utilizes the wurst package to unpack the database, explode it to a list of all exchanges, and save this data 
in a DataFrame as a .pickle binary file.
\index{ExplodeDatabase() (in module WasteAndMaterialFootprint.ExplodeDatabase)@\spxentry{ExplodeDatabase()}\spxextra{in module WasteAndMaterialFootprint.ExplodeDatabase}}

\begin{fulllineitems}
\phantomsection\label{\detokenize{WasteAndMaterialFootprint:WasteAndMaterialFootprint.ExplodeDatabase.ExplodeDatabase}}
\pysigstartsignatures
\pysiglinewithargsret{\sphinxcode{\sphinxupquote{WasteAndMaterialFootprint.ExplodeDatabase.}}\sphinxbfcode{\sphinxupquote{ExplodeDatabase}}}{\sphinxparam{\DUrole{n}{db\_name}}}{}
\pysigstopsignatures
\sphinxAtStartPar
Explode a Brightway2 database into a single\sphinxhyphen{}level list of all exchanges using wurst.
\begin{quote}\begin{description}
\sphinxlineitem{Parameters}
\sphinxAtStartPar
\sphinxstyleliteralstrong{\sphinxupquote{db\_name}} (\sphinxstyleliteralemphasis{\sphinxupquote{str}}) \textendash{} Name of the Brightway2 database to be exploded.

\sphinxlineitem{Returns}
\sphinxAtStartPar
None
The function saves the output to a file and logs the operation, but does not return any value.

\sphinxlineitem{Return type}
\sphinxAtStartPar
None

\end{description}\end{quote}

\end{fulllineitems}



\subsection{WasteAndMaterialFootprint.FutureScenarios module}
\label{\detokenize{WasteAndMaterialFootprint:module-WasteAndMaterialFootprint.FutureScenarios}}\label{\detokenize{WasteAndMaterialFootprint:wasteandmaterialfootprint-futurescenarios-module}}\index{module@\spxentry{module}!WasteAndMaterialFootprint.FutureScenarios@\spxentry{WasteAndMaterialFootprint.FutureScenarios}}\index{WasteAndMaterialFootprint.FutureScenarios@\spxentry{WasteAndMaterialFootprint.FutureScenarios}!module@\spxentry{module}}

\subsubsection{FutureScenarios Module}
\label{\detokenize{WasteAndMaterialFootprint:futurescenarios-module}}
\sphinxAtStartPar
This module is responsible for creating future databases with premise.
\index{FutureScenarios() (in module WasteAndMaterialFootprint.FutureScenarios)@\spxentry{FutureScenarios()}\spxextra{in module WasteAndMaterialFootprint.FutureScenarios}}

\begin{fulllineitems}
\phantomsection\label{\detokenize{WasteAndMaterialFootprint:WasteAndMaterialFootprint.FutureScenarios.FutureScenarios}}
\pysigstartsignatures
\pysiglinewithargsret{\sphinxcode{\sphinxupquote{WasteAndMaterialFootprint.FutureScenarios.}}\sphinxbfcode{\sphinxupquote{FutureScenarios}}}{\sphinxparam{\DUrole{n}{scenario\_list}}}{}
\pysigstopsignatures
\sphinxAtStartPar
Create future databases with premise.

\sphinxAtStartPar
This function processes scenarios and creates new databases based on the premise module. It configures and uses user\sphinxhyphen{}defined settings and parameters for database creation and scenario processing.
\begin{quote}\begin{description}
\sphinxlineitem{Returns}
\sphinxAtStartPar
None
The function does not return any value but performs operations to create and configure databases in Brightway2 based on specified scenarios.

\sphinxlineitem{Raises}
\sphinxAtStartPar
\sphinxstyleliteralstrong{\sphinxupquote{Exception}} \textendash{} If an error occurs during the processing of scenarios or database creation.

\end{description}\end{quote}

\end{fulllineitems}

\index{MakeFutureScenarios() (in module WasteAndMaterialFootprint.FutureScenarios)@\spxentry{MakeFutureScenarios()}\spxextra{in module WasteAndMaterialFootprint.FutureScenarios}}

\begin{fulllineitems}
\phantomsection\label{\detokenize{WasteAndMaterialFootprint:WasteAndMaterialFootprint.FutureScenarios.MakeFutureScenarios}}
\pysigstartsignatures
\pysiglinewithargsret{\sphinxcode{\sphinxupquote{WasteAndMaterialFootprint.FutureScenarios.}}\sphinxbfcode{\sphinxupquote{MakeFutureScenarios}}}{}{}
\pysigstopsignatures
\sphinxAtStartPar
Main function to run the FutureScenarios module.
Only activated if \sphinxtitleref{use\_premise} is set to True in \sphinxtitleref{user\_settings.py}.

\sphinxAtStartPar
Calls the \sphinxtitleref{FutureScenarios} function to create new databases based on the list of scenarios and settings specified in \sphinxtitleref{user\_settings.py}.

\end{fulllineitems}

\index{check\_existing() (in module WasteAndMaterialFootprint.FutureScenarios)@\spxentry{check\_existing()}\spxextra{in module WasteAndMaterialFootprint.FutureScenarios}}

\begin{fulllineitems}
\phantomsection\label{\detokenize{WasteAndMaterialFootprint:WasteAndMaterialFootprint.FutureScenarios.check_existing}}
\pysigstartsignatures
\pysiglinewithargsret{\sphinxcode{\sphinxupquote{WasteAndMaterialFootprint.FutureScenarios.}}\sphinxbfcode{\sphinxupquote{check\_existing}}}{\sphinxparam{\DUrole{n}{desired\_scenarios}}}{}
\pysigstopsignatures
\sphinxAtStartPar
Check the project to see if the desired scenarios already exist, and if so, remove them from the list of scenarios to be created.
Quite useful when running many scenarios, as it can take a long time to create them all, sometimes crashes, etc.

\sphinxAtStartPar
args: desired\_scenarios (list): list of dictionaries with scenario details

\sphinxAtStartPar
returns: new\_scenarios (list): list of dictionaries with scenario details that do not already exist in the project

\end{fulllineitems}

\index{grouper() (in module WasteAndMaterialFootprint.FutureScenarios)@\spxentry{grouper()}\spxextra{in module WasteAndMaterialFootprint.FutureScenarios}}

\begin{fulllineitems}
\phantomsection\label{\detokenize{WasteAndMaterialFootprint:WasteAndMaterialFootprint.FutureScenarios.grouper}}
\pysigstartsignatures
\pysiglinewithargsret{\sphinxcode{\sphinxupquote{WasteAndMaterialFootprint.FutureScenarios.}}\sphinxbfcode{\sphinxupquote{grouper}}}{\sphinxparam{\DUrole{n}{iterable}}\sphinxparamcomma \sphinxparam{\DUrole{n}{n}}\sphinxparamcomma \sphinxparam{\DUrole{n}{fillvalue}\DUrole{o}{=}\DUrole{default_value}{None}}}{}
\pysigstopsignatures
\end{fulllineitems}

\index{make\_possible\_scenario\_list() (in module WasteAndMaterialFootprint.FutureScenarios)@\spxentry{make\_possible\_scenario\_list()}\spxextra{in module WasteAndMaterialFootprint.FutureScenarios}}

\begin{fulllineitems}
\phantomsection\label{\detokenize{WasteAndMaterialFootprint:WasteAndMaterialFootprint.FutureScenarios.make_possible_scenario_list}}
\pysigstartsignatures
\pysiglinewithargsret{\sphinxcode{\sphinxupquote{WasteAndMaterialFootprint.FutureScenarios.}}\sphinxbfcode{\sphinxupquote{make\_possible\_scenario\_list}}}{\sphinxparam{\DUrole{n}{filenames}}\sphinxparamcomma \sphinxparam{\DUrole{n}{desired\_scenarios}}\sphinxparamcomma \sphinxparam{\DUrole{n}{years}}}{}
\pysigstopsignatures
\sphinxAtStartPar
Make a list of dictionaries with scenario details based on the available scenarios and the desired scenarios.

\sphinxAtStartPar
args: filenames (list): list of filenames of available scenarios
desired\_scenarios (list): list of dictionaries with scenario details
years (list): list of years to be used

\sphinxAtStartPar
returns: scenarios (list): list of dictionaries with scenario details that are available and desired

\end{fulllineitems}



\subsection{WasteAndMaterialFootprint.MakeCustomDatabase module}
\label{\detokenize{WasteAndMaterialFootprint:module-WasteAndMaterialFootprint.MakeCustomDatabase}}\label{\detokenize{WasteAndMaterialFootprint:wasteandmaterialfootprint-makecustomdatabase-module}}\index{module@\spxentry{module}!WasteAndMaterialFootprint.MakeCustomDatabase@\spxentry{WasteAndMaterialFootprint.MakeCustomDatabase}}\index{WasteAndMaterialFootprint.MakeCustomDatabase@\spxentry{WasteAndMaterialFootprint.MakeCustomDatabase}!module@\spxentry{module}}

\subsubsection{MakeCustomDatabase Module}
\label{\detokenize{WasteAndMaterialFootprint:makecustomdatabase-module}}
\sphinxAtStartPar
This module contains functions for creating an xlsx representation of a Brightway2 database 
and importing it into Brightway2.

\sphinxAtStartPar
Main functions:
\sphinxhyphen{} dbWriteExcel: Creates an xlsx file representing a custom Brightway2 database.
\sphinxhyphen{} dbExcel2BW: Imports the custom database (created by dbWriteExcel) into Brightway2.
\index{dbExcel2BW() (in module WasteAndMaterialFootprint.MakeCustomDatabase)@\spxentry{dbExcel2BW()}\spxextra{in module WasteAndMaterialFootprint.MakeCustomDatabase}}

\begin{fulllineitems}
\phantomsection\label{\detokenize{WasteAndMaterialFootprint:WasteAndMaterialFootprint.MakeCustomDatabase.dbExcel2BW}}
\pysigstartsignatures
\pysiglinewithargsret{\sphinxcode{\sphinxupquote{WasteAndMaterialFootprint.MakeCustomDatabase.}}\sphinxbfcode{\sphinxupquote{dbExcel2BW}}}{}{}
\pysigstopsignatures
\sphinxAtStartPar
Import the custom database (created by dbWriteExcel) into Brightway2.

\sphinxAtStartPar
This function imports a custom Brightway2 database from an Excel file into the Brightway2 software,
making it available for further environmental impact analysis.
\begin{quote}\begin{description}
\sphinxlineitem{Returns}
\sphinxAtStartPar
None

\end{description}\end{quote}

\end{fulllineitems}

\index{dbWriteExcel() (in module WasteAndMaterialFootprint.MakeCustomDatabase)@\spxentry{dbWriteExcel()}\spxextra{in module WasteAndMaterialFootprint.MakeCustomDatabase}}

\begin{fulllineitems}
\phantomsection\label{\detokenize{WasteAndMaterialFootprint:WasteAndMaterialFootprint.MakeCustomDatabase.dbWriteExcel}}
\pysigstartsignatures
\pysiglinewithargsret{\sphinxcode{\sphinxupquote{WasteAndMaterialFootprint.MakeCustomDatabase.}}\sphinxbfcode{\sphinxupquote{dbWriteExcel}}}{}{}
\pysigstopsignatures
\sphinxAtStartPar
Create an xlsx file representing a custom Brightway2 database.

\sphinxAtStartPar
This function generates an Excel file which represents a custom database for Brightway2,
using predefined directory and database settings.
\begin{quote}\begin{description}
\sphinxlineitem{Returns}
\sphinxAtStartPar
Path to the generated xlsx file.

\end{description}\end{quote}

\end{fulllineitems}

\index{determine\_unit\_from\_name() (in module WasteAndMaterialFootprint.MakeCustomDatabase)@\spxentry{determine\_unit\_from\_name()}\spxextra{in module WasteAndMaterialFootprint.MakeCustomDatabase}}

\begin{fulllineitems}
\phantomsection\label{\detokenize{WasteAndMaterialFootprint:WasteAndMaterialFootprint.MakeCustomDatabase.determine_unit_from_name}}
\pysigstartsignatures
\pysiglinewithargsret{\sphinxcode{\sphinxupquote{WasteAndMaterialFootprint.MakeCustomDatabase.}}\sphinxbfcode{\sphinxupquote{determine\_unit\_from\_name}}}{\sphinxparam{\DUrole{n}{name}}}{}
\pysigstopsignatures
\sphinxAtStartPar
Determine the unit based on the name.
\begin{quote}\begin{description}
\sphinxlineitem{Parameters}
\sphinxAtStartPar
\sphinxstyleliteralstrong{\sphinxupquote{name}} \textendash{} The name from which to infer the unit.

\sphinxlineitem{Returns}
\sphinxAtStartPar
The inferred unit as a string.

\end{description}\end{quote}

\end{fulllineitems}

\index{get\_files\_from\_tree() (in module WasteAndMaterialFootprint.MakeCustomDatabase)@\spxentry{get\_files\_from\_tree()}\spxextra{in module WasteAndMaterialFootprint.MakeCustomDatabase}}

\begin{fulllineitems}
\phantomsection\label{\detokenize{WasteAndMaterialFootprint:WasteAndMaterialFootprint.MakeCustomDatabase.get_files_from_tree}}
\pysigstartsignatures
\pysiglinewithargsret{\sphinxcode{\sphinxupquote{WasteAndMaterialFootprint.MakeCustomDatabase.}}\sphinxbfcode{\sphinxupquote{get\_files\_from\_tree}}}{\sphinxparam{\DUrole{n}{dir\_searchmaterial\_results}}\sphinxparamcomma \sphinxparam{\DUrole{n}{dir\_searchwaste\_results}}}{}
\pysigstopsignatures
\sphinxAtStartPar
Collects filenames from the SearchMaterial and SearchWasteResults directories.
\begin{quote}\begin{description}
\sphinxlineitem{Parameters}\begin{itemize}
\item {} 
\sphinxAtStartPar
\sphinxstyleliteralstrong{\sphinxupquote{dir\_searchmaterial\_results}} \textendash{} Directory path for SearchMaterial results.

\item {} 
\sphinxAtStartPar
\sphinxstyleliteralstrong{\sphinxupquote{dir\_searchwaste\_results}} \textendash{} Directory path for SearchWasteResults.

\end{itemize}

\sphinxlineitem{Returns}
\sphinxAtStartPar
Sorted list of filenames.

\end{description}\end{quote}

\end{fulllineitems}



\subsection{WasteAndMaterialFootprint.MethodEditor module}
\label{\detokenize{WasteAndMaterialFootprint:module-WasteAndMaterialFootprint.MethodEditor}}\label{\detokenize{WasteAndMaterialFootprint:wasteandmaterialfootprint-methodeditor-module}}\index{module@\spxentry{module}!WasteAndMaterialFootprint.MethodEditor@\spxentry{WasteAndMaterialFootprint.MethodEditor}}\index{WasteAndMaterialFootprint.MethodEditor@\spxentry{WasteAndMaterialFootprint.MethodEditor}!module@\spxentry{module}}

\subsubsection{MethodEditor Module}
\label{\detokenize{WasteAndMaterialFootprint:methodeditor-module}}
\sphinxAtStartPar
This module provides functions for adding, deleting, and checking methods related to waste and material footprints in a project.


\paragraph{Function Summary:}
\label{\detokenize{WasteAndMaterialFootprint:function-summary}}\begin{itemize}
\item {} 
\sphinxAtStartPar
\sphinxtitleref{AddMethods}: Adds new methods to a project based on a custom biosphere database.

\item {} 
\sphinxAtStartPar
\sphinxtitleref{DeleteMethods}: Removes specific methods from a project, particularly those related to waste and material footprints.

\item {} 
\sphinxAtStartPar
\sphinxtitleref{CheckMethods}: Lists and checks the methods in a project, focusing on those associated with waste and material footprints.

\end{itemize}
\index{AddMethods() (in module WasteAndMaterialFootprint.MethodEditor)@\spxentry{AddMethods()}\spxextra{in module WasteAndMaterialFootprint.MethodEditor}}

\begin{fulllineitems}
\phantomsection\label{\detokenize{WasteAndMaterialFootprint:WasteAndMaterialFootprint.MethodEditor.AddMethods}}
\pysigstartsignatures
\pysiglinewithargsret{\sphinxcode{\sphinxupquote{WasteAndMaterialFootprint.MethodEditor.}}\sphinxbfcode{\sphinxupquote{AddMethods}}}{}{}
\pysigstopsignatures
\sphinxAtStartPar
Add methods to the specified project based on entries in the custom biosphere database.
\begin{quote}\begin{description}
\sphinxlineitem{Parameters}\begin{itemize}
\item {} 
\sphinxAtStartPar
\sphinxstyleliteralstrong{\sphinxupquote{project\_wmf}} \textendash{} Name of the project.

\item {} 
\sphinxAtStartPar
\sphinxstyleliteralstrong{\sphinxupquote{db\_wmf\_name}} \textendash{} Name of the database.

\end{itemize}

\end{description}\end{quote}

\end{fulllineitems}

\index{CheckMethods() (in module WasteAndMaterialFootprint.MethodEditor)@\spxentry{CheckMethods()}\spxextra{in module WasteAndMaterialFootprint.MethodEditor}}

\begin{fulllineitems}
\phantomsection\label{\detokenize{WasteAndMaterialFootprint:WasteAndMaterialFootprint.MethodEditor.CheckMethods}}
\pysigstartsignatures
\pysiglinewithargsret{\sphinxcode{\sphinxupquote{WasteAndMaterialFootprint.MethodEditor.}}\sphinxbfcode{\sphinxupquote{CheckMethods}}}{}{}
\pysigstopsignatures
\sphinxAtStartPar
Check methods associated with the “WasteAndMaterial Footprint” in the specified project.
\begin{quote}\begin{description}
\sphinxlineitem{Parameters}
\sphinxAtStartPar
\sphinxstyleliteralstrong{\sphinxupquote{project\_wmf}} \textendash{} Name of the project.

\end{description}\end{quote}

\end{fulllineitems}

\index{DeleteMethods() (in module WasteAndMaterialFootprint.MethodEditor)@\spxentry{DeleteMethods()}\spxextra{in module WasteAndMaterialFootprint.MethodEditor}}

\begin{fulllineitems}
\phantomsection\label{\detokenize{WasteAndMaterialFootprint:WasteAndMaterialFootprint.MethodEditor.DeleteMethods}}
\pysigstartsignatures
\pysiglinewithargsret{\sphinxcode{\sphinxupquote{WasteAndMaterialFootprint.MethodEditor.}}\sphinxbfcode{\sphinxupquote{DeleteMethods}}}{}{}
\pysigstopsignatures
\sphinxAtStartPar
Delete methods associated with the “WasteAndMaterial Footprint” in the specified project.
\begin{quote}\begin{description}
\sphinxlineitem{Parameters}
\sphinxAtStartPar
\sphinxstyleliteralstrong{\sphinxupquote{project\_wmf}} \textendash{} Name of the project.

\end{description}\end{quote}

\end{fulllineitems}



\subsection{WasteAndMaterialFootprint.SearchMaterial module}
\label{\detokenize{WasteAndMaterialFootprint:module-WasteAndMaterialFootprint.SearchMaterial}}\label{\detokenize{WasteAndMaterialFootprint:wasteandmaterialfootprint-searchmaterial-module}}\index{module@\spxentry{module}!WasteAndMaterialFootprint.SearchMaterial@\spxentry{WasteAndMaterialFootprint.SearchMaterial}}\index{WasteAndMaterialFootprint.SearchMaterial@\spxentry{WasteAndMaterialFootprint.SearchMaterial}!module@\spxentry{module}}

\subsubsection{SearchMaterial Module}
\label{\detokenize{WasteAndMaterialFootprint:searchmaterial-module}}
\sphinxAtStartPar
This script loads data from ‘\textless{}db name\textgreater{}\_exploded.pickle’, runs search queries, 
and produces a CSV to store the results and a log entry. The search queries are 
formatted as dictionaries with fields NAME, CODE, and search terms keywords\_AND, 
keywords\_OR, and keywords\_NOT. These queries are defined in \sphinxtitleref{config/queries\_waste.py}.
\index{SearchMaterial() (in module WasteAndMaterialFootprint.SearchMaterial)@\spxentry{SearchMaterial()}\spxextra{in module WasteAndMaterialFootprint.SearchMaterial}}

\begin{fulllineitems}
\phantomsection\label{\detokenize{WasteAndMaterialFootprint:WasteAndMaterialFootprint.SearchMaterial.SearchMaterial}}
\pysigstartsignatures
\pysiglinewithargsret{\sphinxcode{\sphinxupquote{WasteAndMaterialFootprint.SearchMaterial.}}\sphinxbfcode{\sphinxupquote{SearchMaterial}}}{\sphinxparam{\DUrole{n}{db\_name}}\sphinxparamcomma \sphinxparam{\DUrole{n}{project\_wmf}}}{}
\pysigstopsignatures
\sphinxAtStartPar
Search for materials in a specified database and extract related information.

\sphinxAtStartPar
This function takes a database name as input, sets the project to the respective database,
and looks for activities involving a predefined list of materials. It extracts relevant details
of these activities, such as ISIC and CPC classifications, and saves the details to a CSV file.
It also extracts related material exchanges and saves them to another CSV file.
\begin{quote}\begin{description}
\sphinxlineitem{Parameters}\begin{itemize}
\item {} 
\sphinxAtStartPar
\sphinxstyleliteralstrong{\sphinxupquote{db\_name}} \textendash{} The name of the database to search in.

\item {} 
\sphinxAtStartPar
\sphinxstyleliteralstrong{\sphinxupquote{project\_wmf}} \textendash{} The Brightway2 project to set as current for the search.

\end{itemize}

\sphinxlineitem{Returns}
\sphinxAtStartPar
None

\sphinxlineitem{Raises}
\sphinxAtStartPar
\sphinxstyleliteralstrong{\sphinxupquote{Exception}} \textendash{} If there is any error in reading the materials list from the file.

\end{description}\end{quote}

\end{fulllineitems}



\subsection{WasteAndMaterialFootprint.SearchWaste module}
\label{\detokenize{WasteAndMaterialFootprint:module-WasteAndMaterialFootprint.SearchWaste}}\label{\detokenize{WasteAndMaterialFootprint:wasteandmaterialfootprint-searchwaste-module}}\index{module@\spxentry{module}!WasteAndMaterialFootprint.SearchWaste@\spxentry{WasteAndMaterialFootprint.SearchWaste}}\index{WasteAndMaterialFootprint.SearchWaste@\spxentry{WasteAndMaterialFootprint.SearchWaste}!module@\spxentry{module}}

\subsubsection{SearchWaste Module}
\label{\detokenize{WasteAndMaterialFootprint:searchwaste-module}}
\sphinxAtStartPar
This script loads data from ‘\textless{}db name\textgreater{}\_exploded.pickle’, runs search queries, 
and produces CSV files to store the results and a log entry. The search queries are 
formatted as dictionaries with fields NAME, CODE, and search terms keywords\_AND, 
keywords\_OR, and keywords\_NOT. These queries are defined in \sphinxtitleref{config/queries\_waste.py}.


\paragraph{Functionality}
\label{\detokenize{WasteAndMaterialFootprint:functionality}}
\sphinxAtStartPar
Provides a function, {\hyperref[\detokenize{WasteAndMaterialFootprint:WasteAndMaterialFootprint.SearchWaste.SearchWaste}]{\sphinxcrossref{\sphinxcode{\sphinxupquote{SearchWaste()}}}}}, that loads data from ‘\textless{}db name\textgreater{}\_exploded.pickle’,
runs search queries, and produces result CSVs and log entries.
\index{SearchWaste() (in module WasteAndMaterialFootprint.SearchWaste)@\spxentry{SearchWaste()}\spxextra{in module WasteAndMaterialFootprint.SearchWaste}}

\begin{fulllineitems}
\phantomsection\label{\detokenize{WasteAndMaterialFootprint:WasteAndMaterialFootprint.SearchWaste.SearchWaste}}
\pysigstartsignatures
\pysiglinewithargsret{\sphinxcode{\sphinxupquote{WasteAndMaterialFootprint.SearchWaste.}}\sphinxbfcode{\sphinxupquote{SearchWaste}}}{\sphinxparam{\DUrole{n}{db\_name}}}{}
\pysigstopsignatures
\sphinxAtStartPar
Load data from ‘\textless{}db name\textgreater{}\_exploded.pickle’, run search queries, and produce
result CSVs and log entries.

\sphinxAtStartPar
This function processes waste\sphinxhyphen{}related data from a given database and runs
predefined queries to identify relevant waste exchanges. The results are
saved in CSV files and log entries are created for each search operation.
\begin{quote}\begin{description}
\sphinxlineitem{Parameters}
\sphinxAtStartPar
\sphinxstyleliteralstrong{\sphinxupquote{db\_name}} (\sphinxstyleliteralemphasis{\sphinxupquote{str}}) \textendash{} The database name to be used in the search operation.

\end{description}\end{quote}

\sphinxAtStartPar
Note:
The queries are defined in \sphinxtitleref{config/queries\_waste.py}.

\end{fulllineitems}



\subsection{WasteAndMaterialFootprint.VerifyDatabase module}
\label{\detokenize{WasteAndMaterialFootprint:module-WasteAndMaterialFootprint.VerifyDatabase}}\label{\detokenize{WasteAndMaterialFootprint:wasteandmaterialfootprint-verifydatabase-module}}\index{module@\spxentry{module}!WasteAndMaterialFootprint.VerifyDatabase@\spxentry{WasteAndMaterialFootprint.VerifyDatabase}}\index{WasteAndMaterialFootprint.VerifyDatabase@\spxentry{WasteAndMaterialFootprint.VerifyDatabase}!module@\spxentry{module}}

\subsubsection{VerifyDatabase Module}
\label{\detokenize{WasteAndMaterialFootprint:verifydatabase-module}}
\sphinxAtStartPar
This module contains a function to verify a (WasteAndMaterialFootprint) database 
within a given project in Brightway2. It performs a verification by calculating LCA scores 
for random activities within the specified database using selected methods.
\index{VerifyDatabase() (in module WasteAndMaterialFootprint.VerifyDatabase)@\spxentry{VerifyDatabase()}\spxextra{in module WasteAndMaterialFootprint.VerifyDatabase}}

\begin{fulllineitems}
\phantomsection\label{\detokenize{WasteAndMaterialFootprint:WasteAndMaterialFootprint.VerifyDatabase.VerifyDatabase}}
\pysigstartsignatures
\pysiglinewithargsret{\sphinxcode{\sphinxupquote{WasteAndMaterialFootprint.VerifyDatabase.}}\sphinxbfcode{\sphinxupquote{VerifyDatabase}}}{\sphinxparam{\DUrole{n}{project\_name}}\sphinxparamcomma \sphinxparam{\DUrole{n}{database\_name}}\sphinxparamcomma \sphinxparam{\DUrole{n}{check\_material}\DUrole{o}{=}\DUrole{default_value}{True}}\sphinxparamcomma \sphinxparam{\DUrole{n}{check\_waste}\DUrole{o}{=}\DUrole{default_value}{True}}\sphinxparamcomma \sphinxparam{\DUrole{n}{log}\DUrole{o}{=}\DUrole{default_value}{True}}}{}
\pysigstopsignatures
\sphinxAtStartPar
Verifies a database within a given project in Brightway2 by calculating LCA scores
for random activities using selected methods.

\sphinxAtStartPar
This function assesses the integrity and validity of a specified database within a Brightway2 project.
It performs LCA calculations on random activities using Waste Footprint and Material Demand Footprint methods,
and logs the results.
\begin{quote}\begin{description}
\sphinxlineitem{Parameters}\begin{itemize}
\item {} 
\sphinxAtStartPar
\sphinxstyleliteralstrong{\sphinxupquote{project\_name}} (\sphinxstyleliteralemphasis{\sphinxupquote{str}}) \textendash{} The name of the Brightway2 project.

\item {} 
\sphinxAtStartPar
\sphinxstyleliteralstrong{\sphinxupquote{database\_name}} (\sphinxstyleliteralemphasis{\sphinxupquote{str}}) \textendash{} The name of the database to be verified.

\item {} 
\sphinxAtStartPar
\sphinxstyleliteralstrong{\sphinxupquote{check\_material}} (\sphinxstyleliteralemphasis{\sphinxupquote{bool}}) \textendash{} If True, checks for Material Demand Footprint methods.

\item {} 
\sphinxAtStartPar
\sphinxstyleliteralstrong{\sphinxupquote{check\_waste}} (\sphinxstyleliteralemphasis{\sphinxupquote{bool}}) \textendash{} If True, checks for Waste Footprint methods.

\item {} 
\sphinxAtStartPar
\sphinxstyleliteralstrong{\sphinxupquote{log}} (\sphinxstyleliteralemphasis{\sphinxupquote{bool}}) \textendash{} If True, logs the results.

\end{itemize}

\sphinxlineitem{Returns}
\sphinxAtStartPar
Exit code (0 for success, 1 for failure).

\end{description}\end{quote}

\end{fulllineitems}



\chapter{Indices and tables}
\label{\detokenize{index:indices-and-tables}}\begin{itemize}
\item {} 
\sphinxAtStartPar
\DUrole{xref,std,std-ref}{genindex}

\item {} 
\sphinxAtStartPar
\DUrole{xref,std,std-ref}{modindex}

\item {} 
\sphinxAtStartPar
\DUrole{xref,std,std-ref}{search}

\end{itemize}


\renewcommand{\indexname}{Python Module Index}
\begin{sphinxtheindex}
\let\bigletter\sphinxstyleindexlettergroup
\bigletter{w}
\item\relax\sphinxstyleindexentry{WasteAndMaterialFootprint.ExchangeEditor}\sphinxstyleindexpageref{WasteAndMaterialFootprint:\detokenize{module-WasteAndMaterialFootprint.ExchangeEditor}}
\item\relax\sphinxstyleindexentry{WasteAndMaterialFootprint.ExplodeDatabase}\sphinxstyleindexpageref{WasteAndMaterialFootprint:\detokenize{module-WasteAndMaterialFootprint.ExplodeDatabase}}
\item\relax\sphinxstyleindexentry{WasteAndMaterialFootprint.FutureScenarios}\sphinxstyleindexpageref{WasteAndMaterialFootprint:\detokenize{module-WasteAndMaterialFootprint.FutureScenarios}}
\item\relax\sphinxstyleindexentry{WasteAndMaterialFootprint.main}\sphinxstyleindexpageref{WasteAndMaterialFootprint:\detokenize{module-WasteAndMaterialFootprint.main}}
\item\relax\sphinxstyleindexentry{WasteAndMaterialFootprint.MakeCustomDatabase}\sphinxstyleindexpageref{WasteAndMaterialFootprint:\detokenize{module-WasteAndMaterialFootprint.MakeCustomDatabase}}
\item\relax\sphinxstyleindexentry{WasteAndMaterialFootprint.MethodEditor}\sphinxstyleindexpageref{WasteAndMaterialFootprint:\detokenize{module-WasteAndMaterialFootprint.MethodEditor}}
\item\relax\sphinxstyleindexentry{WasteAndMaterialFootprint.SearchMaterial}\sphinxstyleindexpageref{WasteAndMaterialFootprint:\detokenize{module-WasteAndMaterialFootprint.SearchMaterial}}
\item\relax\sphinxstyleindexentry{WasteAndMaterialFootprint.SearchWaste}\sphinxstyleindexpageref{WasteAndMaterialFootprint:\detokenize{module-WasteAndMaterialFootprint.SearchWaste}}
\item\relax\sphinxstyleindexentry{WasteAndMaterialFootprint.VerifyDatabase}\sphinxstyleindexpageref{WasteAndMaterialFootprint:\detokenize{module-WasteAndMaterialFootprint.VerifyDatabase}}
\end{sphinxtheindex}

\renewcommand{\indexname}{Index}
\printindex
\end{document}